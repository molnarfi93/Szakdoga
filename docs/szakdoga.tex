\documentclass[a4paper,12pt]{article}

% Set margins
\usepackage[hmargin=2.5cm, vmargin=3cm]{geometry}

\frenchspacing

% Language packages
\usepackage[utf8]{inputenc}
\usepackage[T1]{fontenc}
\usepackage[magyar]{babel}

% AMS
\usepackage{amssymb,amsmath}

% Graphic packages
\usepackage{graphicx}

% Colors
\usepackage{color}
\usepackage[usenames,dvipsnames]{xcolor}

% Enumeration
\usepackage{enumitem}

% Links
\usepackage{hyperref}

% Question
\newenvironment{question}[1]
{\noindent\textcolor{OliveGreen}{$\circ$ \textit{#1}}

\smallskip

\color{Gray}

}{\bigskip}

% Task
\newenvironment{task}[1]
{\noindent\textcolor{RoyalBlue}{$\circ$ \textit{#1}}

\smallskip

\color{Gray}

}{\bigskip}

% Notification
\newenvironment{notification}[1]
{\noindent\textcolor{Peach}{$\circ$ \textit{#1}}

\smallskip

\color{Gray}

}{\bigskip}

% Problem
\newenvironment{problem}[1]
{\noindent\textcolor{OrangeRed}{$\circ$ \textit{#1}}

\smallskip

\color{Gray}

}{\bigskip}

% Solution
\newenvironment{solution}
{\noindent\color{Violet}}{\bigskip}

% Starred
\newenvironment{starred}
{\noindent\color{Maroon}}{\bigskip}

\begin{document}

\begin{center}
    \huge \textbf{Órarendgenerálási probléma megoldási lehetőségei} 
\end{center}

Az iskolák és egyetemek a különböző osztályok, tanárok és tantermek órarendjének 
elkészítéséhez ma már számítógépes segítséget használnak, hagyományos algoritmusokat és a
mesterséges intelligencia egyik elterjedt eszközét, a genetikus algoritmust. Máskülönben 
nagyon időigényes, fáradtságos munka lenne összesen többszáz órarend ütközésmentes 
összehangolása, hiszen ugyanannak az osztálynak egy időben nem lehet több tanórája, egy tanár
sem tud két helyen egyszerre jelen lenni és egy tanteremben sem lehet két tanóra egy időben,
ott van továbbá a termek kapacitása, a tanárok heti óraszámai közti különbségek 
minimalizálása, az új osztályok képzésének szükségessége, pl. nyelvi vagy fakultációs órák 
miatt, és ez még nem minden. Munkám során olyan algoritmusokat írok, melyeknek köszönhetően 
adathalmaz (osztályok, tanárok, tantermek, tantárgyak) megadása után ütközésmentes és minden
feltételnek eleget tevő órarendeket tudunk generálni, majd grafikus felhasználói felületet 
hozok létre fölötte, mégpedig webeset.

Számos erre a célre létrehozott szoftver készült már, ingyenes/fizetős, platformfüggetlen/
valamely platformra létrehozott, interaktív/automatikus. Az interaktív használat itt azt
jelenti, hogy mi adunk meg órarendet, melynek ütközésmentességét leteszteljük a programmal,
ami jelzésünkre adja, mi mivel ütközik és milyen lehetőségeink vannak a kiküszöbölésre. 
A legnépszerűbb applikációkat hasonlítom össze:
-TimeTabler: A legrégebbi, mai napig nagyon népszerű órarendütemező, olyan extra funkciókkal,
             mint a részmunkaidős tanárok felvétele a rendszerbe és a választható
             tanóra-időtartamok (szimpla vagy dupla órák). Interaktív módon is használható. 
             Ingyenesen letölthető egy demo verzió nem megváltoztatható, demo adatokkal és 
             tutorial videóval. Előfizetés nincs, csak megvásárlás lehetséges. 
             Platformfüggetlen, telefonos verzióval viszont nem rendelkezik.
-Schedule Builder: Ingyenes, webes alkalmazás. Csak az alapfunkciókkal rendelkezik,
                   nincsenek extrák, illetve nem elérhető az automatikus használat sem.
                   Platformfüggetlen.
-Prime TimeTable: Macintosh, Linux, Android és IOS rendszereken egyaránt elérhető, viszont
                  Windowson nem. Rendelkezik ugyanazokkal a funkciókkal, mint a TimeTabler,
                  beleértve az interaktív használatot is. 30 napos próbaverzióval vagy éves
                  előfizetéssel lehet használni.
-aSc TimeTables: A legjobbnak választott órarendgeneráló applikáció. Interaktív és
                 automatikus használat, extra funkciók, próbaverzió. Előfizetés nincs, csak
                 megvásárlás lehetséges, használata Windowsra és Macintoshra korlátozódik.
Mint látható, mindegyik fizetős applikáció rendelkezik automatikus és interaktív használati
lehetőséggel, valamint az extra funkciókkal. A nyomtatási és exportálási lehetőségekkel
pedig az ingyenes Schedule Builder is. 

Nagyrészt a Guido van Rossum, holland programozó által alkotott Python multiparadigmás 
(vagyis a procedurális, objektumorientált, funkcionális programozást egyaránt támogató)
programnyelvet, annak 3.8.2-es verzióját használtam. Jellemzői még az interpreteres 
feldolgozás (a futtatás közvetlenül a forráskódból történik) és a platformfüggetlenség 
(mindegyik széles körben elterjedt operációs rendszeren használható). Pythonra épülő
ismertebb alkalmazások a Plone tartalomkezelő rendszer, a Blender animációs modellező
és a Torrent fájlmegosztó. Python nyelvben a tömbök alapértelmezetten dinamikusak, vagyis
listák, valamint a szótár típusú változó is rendelkezésre áll ebben a nyelvben, így 
kényelmes volt az adatkezelés. Utóbbi adattípus abban különbözik a listától, hogy az 
elemei rendelkeznek azonosító névvel vagy kóddal, ennek megadásával lehet rájuk hivatkozni, 
nem tömbindexszel, ami különböző típusú elemek esetén hasznos. A szemétgyűjtés 
automatikusan történik, nincsen szükség memóriafelszabadításra és memóriafoglalásra az 
algoritmus során kiszelektálódó, illetve létrejövő egyedek kapcsán, bár mint kiderült,
erre nem volt szükségem.

A Brendon Eich által kifejlesztett JavaScript (JS) a webprogramozásban nagyon elterjedt
kliensoldali szkriptnyelv, nehéz megkerülni, mert a HTML csak egy jelölőnyelv. A JavaScript 
biztosítja a weboldalak interaktivitását is, mert alkalmas az eseményvezérelt programozásra. 
Multiparadigmás, just-in-time (JIT) compileres. Olyan kiegészítő technológiák, mint a JSON, 
a Jquery vagy az AJAX, mind elérhetőek beépítetten, továbbá olyan keretrendszereket is 
használhatunk, mint a Bootstrap vagy a Vue. Ez utóbbi nagyon hasznos volt a számomra, a 
felhasználói interfészek létrehozásával kapcsolatban, a Vue jobb altenatívát adott az 
űrlapokkal történő megoldás helyett. Az egyszerűség kedvéért nem is HTML, hanem Vue fájlokat
készítettem, bennük HMTL kódrésszel. Önmagában a JavaScript-et pedig a router 
(útvonalválasztó) fájl létrehozásához használtam.

Donald D. Chamberlin SQL nevű adatbáziskezelő nyelve, illetve Richard Hipp SQLite 
adatbáziskezelő rendszere az, amelyet az adatok letárolására és kezelésére (lekérdezésére, 
módosítására), a háttérben használtam, a webfejlesztésben érdekelt komolyabb vállalatok 
többségéhez hasonlóan. Az SQL parancsnyelv relációs adatbázisok kezelésére lett létrehozva, 
az objektum-relációs leképezéshez pedig Python esetén SQLAlchemy-t használunk, így én is ezt 
tettem. Erre azért van szükség, mert ha objektum-orientált paradigmát követünk, az adatbázis 
relációs struktúrája ugyan képes minden szükséges információt megfelelő formában rögzíteni, 
ezenfelül a viszonyok objektum-orientált feldolgozása szükséges. Az SQLAlchemy automatikus 
sessionkezelést biztosít, ami jelentősen megkönnyíti a munkát, főleg a felhasználókezeléssel 
kapcsolatban.

Charles Darwin, angol biológus evolúcióelmélete paradigmaváltást hozott magával a biológiában. 
Lényege a fajok fennmaradásának, túlélésének genetikai háttére a Föld ökoszisztémáiban 
(életközösségeiben). Összetevői a szelekció, az öröklődés és a mutáció. A természetes 
szelekció értelmében a gyenge, vagyis kedvezőtlenebb, kevésbé optimális tulajdonságokkal 
rendelkező egyedeknek el kell pusztulniuk annak érdekében, hogy az adott faj populációja 
életképesebb legyen a következő generációkban. A kedvezőtlen gének továbbörökítése ezt
hátráltatná és mivel az állatvilágban táplálkozási hierarchia van, továbbá az éghajlati 
körülményekhez való alkalmazkodás is szerepet játszik a túlélésben, ebben az esetben félő 
lenne a populáció kihalása. A nőstények ösztönösen a kedvezőbb tulajdonságokkal rendelkező 
hímeket választják párosodásra, vagypedig egymás elleni csatában lejátsszák a hímek, ki 
juthasson hozzá a nőstényhez. Ez lehet nem halálos vagy halálos játszma is. Szintén kegyetlen 
világ figyelhető meg pl. a fókák esetében, ahol a győztes hím az összes nősténnyel párosodik, 
a vesztes pedig eggyel sem. Többnyire az erősebb egyedek örökítik így tovább a génjeiket, a
gyengébbek fokozatosan elhullanak a populációból, amely így generációról generációra erősebb,
életképesebb lesz. A mutáció szerepe is fontos, melynek eredménye hogy néha-néha egy egyed
valamely génje egyik szülőtől sem származó lesz, hanem egészen más lesz, következményként
nagymértékben különbözni fog mindkét szülőtől valamely tulajdonság kapcsán. Ez a jelenség is 
jó a populáció számára, mert csökkenti a belterjesség mértékét, a korai generációk folyamán
bekövetkező mutációk segíthetnek az adott ökoszisztémában hasznos, korábban nem jellemző
tulajdonságok térnyerésében, a későbbi mutációk már hamar elhalnak, nem játszanak szerepet.

A genetikus algoritmusokat olyan diszkrét optimalizálási problémák megoldására alkalmazzuk,
amelyekre nem létezik általános eljárás, a probléma specifikussága, egyedisége következtében.
Túl sok az ismeretlen és a közöttük levő összefüggések is ismeretlenek, így a rendszernek 
"tanulnia kell" a probléma megoldása során, vagyis mesterséges intelligenciát kell 
használnunk. A genetikus algoritmus épp megfelel. Ahogyan az evolúció is lassú folyamat volt,
úgy az evolúciót modellező genetikus algoritmusok is lassúak, ahhoz képest hogy áramkörökről
beszélhetünk a háttérben, de nyilván nagyságrendekkel hamarabb kapunk eredményt, mintha emberi
intelligenciával dolgoznánk és amennyiben van megoldás, akkor azt egy megfelelően elkészített
genetikus algoritmus biztosan megtalálja. Mivel megelőzően nem lehetünk biztosak benne, hogy
egészen megfelelő paramétereket választottunk, az algoritmus többszöri lefuttatásával 
szerezhetünk bizonyosságot a megoldás optimális értékeit illetően. Ezenkívül van rá lehetőség,
hogy az algoritmus párhuzamosításával növeljük a gyorsaságot és csökkentsük a program méretét,
ezáltal a memóriaigényt.
Ahogyan az evolúció törvényei következtében kitermelődtek az egyes fajok optimális 
tulajdonságokkal rendelkező egyedekből álló populációi, a túlélés érdekében, úgy termeljük ki
egy adathalmaz optimális értékekkel rendelkező tagjait az evolúció mozzanatait (szelekció,
öröklődés, mutáció) a maga módján leutánzó genetikus algoritmussal, az adott probléma 
megoldása érdekében. Ahogyan más területeken, úgy itt is leképezzük a valós világot a
programozás nyelvére, nézzük az analógiákat általánosságban:
-populáció->adathalmaz
-egyed->adathalmaz egy tagja
-gén->adathalmaz egy tagjának egy attribútuma
-túléléshez szükséges tulajdonságok->adathalmaz tagjainak attribútumai
-ezen tulajdonságok kedvező mivolta->az attribútumok optimális értéke
A szelekció, öröklődés és mutáció függvények segítségével valósul meg, de ezekkel
kapcsolatban többféle lehetőség létezik és az adott problémánk ismeretében választjuk ki,
hogy milyen fajta szelekciós, öröklődési, illetve mutációs eljárást alkalmazunk. Ezekre nem
térek ki, csak az én munkám során használt eljárásokat fogom bemutatni.
A genetikus algoritmus esetleges sikertelenségét a lokális optimumok problematikája okozza. 
Azt képzeljük el, hogy egy hegyvidék legalacsonyabb völgyét kell megtalálnunk, de miután a
genetikus algoritmus talál egy völgyet (vagyis egy lokális minimumot), nem tudja megmondani,
hogy a legmélyebbet (vagyis egyben a globális optimumot) találta-e meg. A programozó feladata
annak elérése, hogy a rendszer ne gondolja azt tévesen, a globális optimumot találta meg. 
Máskülönben megreked az algoritmus, amely nem szeretne a völgyből megindulni felfelé, hiszen 
ekkor távolodunk az optimális megoldástól. Csakhogy ez az optimális megoldás könnyen lehet,
hogy csak lokális optimum, ami nekünk nem elég. Ahhoz, hogy a globális optimumot megtaláljuk,
átmenetileg el kell távolodnunk a már viszonylag jó megoldástól. Nézzük, miként tudjuk
módosítani a paramétereket a lokális minimumon való megrekedés elkerülése érdekében:
-nagyobb populációval
-a keresztezések és mutációk valószínűségének növelésével
-generációnként kevesebb egyed kiszelektálásával
-de az is előfordulhat, hogy hibát követtünk el a probléma lemodellezése során.
Ezenkívül egyes, rendkívüli összetettségű problémák esetén lehetséges és érdemes több, 
egymással kapcsolatban álló populációt létrehozni.

Van egy eredeti problémánk, melyben tényezőink a tantárgyak, tantermek, tanárok és 
osztályok. Mindenképpen szükség van ennek a felbontására, hogy egyszerűbb részproblámákat kapjunk, melyek megoldhatóak hagyományos algoritmusokkal, illetve hagyományos optimalizálási módszerekkel és együttesen alkotják az órarendgenerálási probléma megoldását. A genetikus algoritmust a plusz feltételeknek lehető legjobban megfelelő órarendek megkapásához fogjuk használni, merthogy ez hagyományos módszerekkel nem megoldható. Ezt onnan lehet tudni, hogy nem tudjuk matematikailag leírni a szükséges algoritmust. A plusz feltételek azok a feltételek, melyek nélkül is érvényes és elfogadható, de nem feltétlenül kielégítő órarendeket kapunk. Esetünkben a kora reggeli/késő délutáni órák számának minimalizálása és a heti óraszámok arányos eloszlása/pénteki órák számának minimalizálása lesznek a plusz feltételek, de lehetne mondjuk a lyukas órák számának minimalizálása is.

\section{Teremhozzárendelési feladat}

Az első részfeladat a teremhozzárendelési, amely kétfázisú lesz. Az első fázisban osztályokat kell hozzárendelni optimálisan a tantermekhez. Ez egy halmazfelbontási feladat, ahogyan az optimalizálásban nevezzük, ami arról szól, hogy az egyik halmaz (osztályok) és másik halmaz (tantermek) elemeihez 1:1 hozzárendelést kell elvégezni, vagyis minden osztályhoz pontosan
1 tanteremnek kell tartoznia és minden tanteremhez pontosan 1 osztálynak. Mindezt úgy, hogy a legoptimálisabb megoldást kapjuk a kihasználatlanság minimalizálásának szempontjából, mert így elkerülhetjük azt, hogy egy osztály terem nélkül maradjon, miközben létezik olyan hozzárendelés, hogy minden osztálynak jusson terem. A kihasználatlanság az adott terem kapacitásának és az adott osztály létszámának a különbsége. Természetesen amennyiben van olyan osztály, amelynek létszáma nagyobb, mint bármely terem befogadóképessége, vagy több olyan, adott létszámot elérő osztály van, mint ahány terem, amelyik rendelkezik az adott befogadóképességgel, akkor a feladat nem megoldható, így legelőször ezzel kapcsolatban kell ellenőrzést végezni. Feltétel nyilván, hogy egy osztálynak nem lehet órája olyan teremben, amelynek a kapacitása kisebb az osztály létszámánál, ezenkívül egy évfolyam osztályai és az évfolyamon képzett nyelvi, fakultációs vagy egyéb csoportok esetében megengedett ugyanannak a teremnek a hozzárendelése. A második
fázis azért szükséges, mert egyes termek lehetnek elsősorban adott tantárgy(ak) számára fenntartottak). Itt már nem osztályokat, hanem osztály-tantárgy kettősöket veszünk (vagyis minden osztályt annyiszor kell számba venni, ahány tantárgy tartozik hozzá), olyan tantárgyak esetén, melyekhez van "speciális" terem/termek, de az első fázis során hozzárendelt terem nem ilyen, akkor az osztály úgymond költözni fog az adott tantárgy kapcsán. Kivéve ha nagyobb a
létszámuk minden ilyen terem befogadóképességénél, ez esetben maradnak az eredetileg
hozzárendelt teremben. 

\noindent \textbf{A feladat formalizálása}

\begin{itemize}
    \item $n \in \Bbb{N}$: osztályok száma
    \item $m \in \Bbb{N}$: tantermek száma
    \item $p \in \Bbb{N}$: tantárgyak száma
    \item $l \in \Bbb{N}^n$: osztályok létszáma
    \item $k \in \Bbb{N}^m$: tantermek kapacitása
    \item $h \in \Bbb{N}^n$: termek száma, ahová az egyes osztályok beférnek
    \item $C \in \Bbb{N}^{n \times m}$: költségmátrix
    \item $A \in \{0;1\}^{n \times m}$: párosításmátrix
    \item $X \in \{0;1\}^{n \times m}$: hozzárendelés-mátrix
\end{itemize}

\[
C_{ij} =
\begin{cases}
k_j-l_i, & \hbox{ha } X_{ij}=1, \\
\infty & \hbox{egyébként}.
\end{cases}
\]

\[
A_{ij} =
\begin{cases}
1, & \hbox{ha } l_i \leq k_j, \\
0& \hbox{egyébként}.
\end{cases}
\]

\[
X_{ij} =
\begin{cases}
1, & \hbox{ha az $i$-edik osztálynak a $j$-edik teremben lesz a tanóra}, \\
0&
\hbox{egyébként}.
\end{cases}
\]

$$\sum_{j=1}^m C_jX_j \rightarrow \hbox{min}$$

$$\sum_{j=1}^m A_{ij}X_j=1$$

$$i=1, 2, \ldots, n$$

$$j=1, 2, \ldots, m$$

$$k=1, 2, \ldots, p$$
Összes lehetséges esetek száma:

$$P=\prod_{i=1}^n h_i$$
Összes lehetséges megoldások száma:

$$P=\prod_{i=1}^n h_i-(i-1)$$
Nézzük az összes lehetséges esetek és lehetséges megoldások számát az alapesetben, az eredeti példa esetén,
illetve évfolyamonként 6, 7, 8 osztályra kibővített esetben (ahol a termek számát plusz egy évfolyamonkénti osztály
esetén 5-tel növeltem): 

$$
\begin{tabular}{|l|c|c|c|c|}
\hline
& 5 & 6 & 7 & 8 \\
\hline
Összes esetek & $1,1 \cdot 10^{23}$ & $1,4 \cdot 10^{31}$ & $1,7 \cdot 10^{38}$ & $2 \cdot 10^{46}$ \\ 
\hline
Összes megoldások & $5,3 \cdot 10^{16}$ & $3,5 \cdot 10^{22}$ & $1,5 \cdot 10^{26}$ & $3,3 \cdot 10^{31}$ \\
\hline
\end{tabular}
$$

A növekedési rend a következő: $T(n,m,p)=\sim \Theta (2nm + n^2p^2).$


\section{Tanárhozzárendelési feladat}

Az osztály-tantárgy kettősökhöz immáron a tanárokat rendeljük hozzá. Ez egy halmazlefedési feladat, ami abban különbözik a halmazfelbontásitól, hogy 1:N hozzárendelés van, ugyanis egy tanárhoz több osztály-tantárgy kettőst is rendelhetünk, sőt ugye kell is többet hozzárendelni. A párosításmátrixban két feltételtől is függ, hogy 0 vagy 1 kerül a rublikába. Egyrészt, hogy az adott tanár tudja-e tanítani az adott tantárgyat, másrészt hogy az adott osztálynak van-e ilyen tárgya.
A minimalizálás pedig itt arra vonatkozik, hogy minél kisebb legyen a tanárok heti óraszámai közötti eltérés, ne fordulhasson elő, hogy mondjuk míg valaki 30 órát tart egy héten, addig más 5-öt.
A futtatás után kapott eredményt látva megállapítható, hogy amennyire lehetett, sikerült kiküszöbölni a tanárok egyenlőtlen terhelését, megkaptuk a lehető legoptimálisabb, olyan hozzárendelés-mátrixot, amely meghatározza, hogy egy adott osztálynak egy adott tantárgyat, melyik tanár tartsa.

\noindent \textbf{A feladat formalizálása}

\begin{itemize}
    \item $n$: osztály-tantárgy kettősök száma
    \item $m$: tanárok száma
    \item $p$: tantárgyak száma
    \item $u \in \Bbb{N}^p$: az egyes tárgyakat hallgató osztályok száma
    \item $v \in \Bbb{N}^p$: az egyes tárgyakat oktató tanárok száma
    \item $t \in \Bbb{N}^m$: tanárok heti óraszáma
    \item $o \in \Bbb{N}^n$: osztály-tantárgy kettősök heti óraszáma
    \item $A \in \{0;1\}^{n \times m}$: párosításmátrix
    \item $X \in \{0;1\}^{n \times m}$: hozzárendelés-mátrix
\end{itemize}

\[
t_{j} =
\begin{cases}
t_j+o_i,& \hbox{ha } X_{ij}=1, \\
t_j & \hbox{egyébként}.
\end{cases}
\]

\[
A_{ij} =
\begin{cases}
1, & \hbox{ha az $i$-edik osztály-tantárgy kettősben szereplő tantárgyat tudja tanítani a $j$-edik tanár} \\
0 & \hbox{egyébként}.
\end{cases}
\]

\[
X_{ij} =
\begin{cases}
1, & \hbox{ha az $i$-edik osztály-tantárgy kettősben szereplő osztálynak az ugyanezen kettősben szereplő tantárgyat a $j$-edik tanár fogja tartani} \\
0 & \hbox{egyébként}.
\end{cases}
\]

$$\sum_{j=1}^m \vert o_jX_j-\overline{o}\vert \rightarrow \hbox{min}$$

$$\sum_{j=1}^m A_{ij} X_j=1$$

Összes lehetséges esetek száma:

$$P=m^n$$
Összes lehetséges megoldások száma:

$$P=\prod_{k=1}^p v_k^{u_k}$$
A lehetséges esetek és megoldások száma (a tanárok számát plusz egy évfolyamonkénti osztály esetén 3-mal növeltem):

$$
\begin{tabular}{|l|c|c|c|c|}
\hline
& 5 & 6 & 7 & 8 \\
\hline
Összes esetek & $9,3 \cdot 10^{325}$ & $6,5 \cdot 10^{395}$ & $7,8 \cdot 10^{467}$ & $9,2 \cdot 10^{541}$ \\
\hline 
Összes megoldások & $1,1 \cdot 10^{132}$ & $2 \cdot 10^{168}$ & $4,2 \cdot 10^{208}$ & $1,2 \cdot 10^{249}$ \\
\hline
\end{tabular}
$$
Növekedési rend: $T(n,m)=\Theta (nm+nm^2).$

$$
\begin{tabular}{|c|c|c|c|}
\hline
5 & 6 & 7 & 8 \\
\hline
$19800$ & $25740$ & $32400$ & $39780$ \\
\hline
\end{tabular}
$$


\section{Időablakok beosztása}

Ennek lényege, hogy a korábbiak alapján kapott osztály-tantárgy-terem-tanár
hozzárendeléseket ütközésmentesen beosszuk időablakokba. Merthogy egy időablakban nyilván nem szerepelhet többször ugyanaz az osztály, ugyanaz a terem és ugyanaz a tanár sem. Egy időablak legfeljebb annyi hozzárendelést tartalmazhat, amennyi a tanárok száma. Amire még oda kell figyelni, hogy egy osztály-tantárgy kettős annyiszor forduljon elő, amennyi a heti óraszáma az adott osztálynak az adott tárgyból.

\noindent \textbf{A feladat formalizálása}

\begin{itemize}
    \item $n \in \Bbb{N}$: osztály-tantárgy kettősök száma
    \item $m \in \Bbb{N}$: tanárok száma
    \item $p \in \Bbb{N}$: tantermek száma
    \item $r \in \Bbb{N}$: időablakok száma
    \item $o \in \Bbb{N}^n$: osztály-tantárgy kettősök heti óraszámai
    \item $c \in \Bbb{N}^n$: osztály-tantárgy kettősök előfordulásainak száma
    \item $e \in \Bbb{N}^m$: tanárok előfordulásainak száma
    \item $h \in \Bbb{N}^p$: tantermek előfordulásainak száma
\end{itemize}

$$\sum_{i=1}^n c_i \in (0,1), \quad \hbox{minden }l\hbox{ időablak esetén.}$$

$$\sum_{j=1}^m e_j \in (0,1), \quad \hbox{minden }l\hbox{ időablak esetén.}$$

$$\sum_{k=1}^p h_k \in (0,1), \quad \hbox{minden }l\hbox{ időablak esetén.}$$

$$\sum_{i=1}^n \vert o_i - c_i \vert = 0$$

$$l=1, 2, \ldots, r$$

Összes lehetséges esetek száma:

$$P=\binom{n+m-1}{m}$$
Összes lehetséges megoldások száma:

$$P=\binom{n}{m}$$
A lehetséges esetek és megoldások száma (a tanárok számát plusz egy évfolyamonkénti osztály esetén 3-mal növeltem):

$$
\begin{tabular}{|l|c|c|c|c|}
\hline
& 5 & 6 & 7 & 8 \\
\hline
Összes esetek & $5 \cdot 10^{105}$ & $9,1 \cdot 10^{119}$ & $1,7 \cdot 10^{134}$ & $3,5 \cdot 10^{148}$ \\
\hline 
Összes megoldások & $8,9 \cdot 10^{36}$ & $6,8 \cdot 10^{41}$ & $4,5 \cdot 10^{46}$ & $2,7 \cdot 10^{51}$ \\
\hline
\end{tabular}
$$
Növekedési rend: $T(n,r,o)=\sim \Theta(nro)$.

$$
\begin{tabular}{|c|c|c|c|}
\hline
5 & 6 & 7 & 8 \\
\hline
$24400$ & $28960$ & $33520$ & $38080$ \\
\hline
\end{tabular}
$$


\section{Időintervallumok hozzárendelése}

A megkapott időablakok még nincsenek időintervallumhoz kötve, bármelyik felcserélhető bármelyikkel. Azt a feladatot, hogy a megadott plusz feltételek alapján lehető legoptimálisabban állapítsuk meg, mely időablak mely időintervallumba kerüljön (vagyis a legoptimálisabb legyen az időablakok sorrendje), a genetikus algoritmus végzi el nekünk.

\noindent A probléma leképezése a genetikus algoritmus összetevőire:
\begin{itemize}
    \item \textbf{gén:} egy konkrét, általános- és középiskolák esetén egy óra hosszúságú, 	          főiskolák/egyetemek esetén két óra hosszúságú időintervallum. Implementációja szótár, két elemmel: nap (enum) és a napon belüli időablak (string).
    \item \textbf{egyed:} az összes időablak sorrendje, az egyedek mibenlétét a gének sorrendje határozza meg. Implementációja szintén szótár, elemei a gének listája, a célfüggvény-érték és az
egyedi azonosító.
    \item \textbf{populáció:} az egyedek összességét tárolja, melyeknek számát a programozó állapítja meg, a hatékonysági tesztek által. Implementációja lista.
    \item \textbf{célfüggvény:} esetemben a kora reggeli/késői délutáni órák számát akarom minimalizálni, illetve a napi óraszámok arányos eloszlása/pénteki órák számának minimalizálása is szempont. Minden tanárhoz tartozik egy \textsl{balance} és egy \textsl{extremisms} érték,
melyet a felhasználó ad meg és ezzel skálán rögzíti mennyire fontosak/kevésbé fontosak ezek a különböző szempontok az adott tanároknak. Ezek az értékek adják a célfüggvény súlyozását, melynek értéke ennek megfelelően minden problémás időintervallum esetén növekszik. Minél kisebb a célfüggvény értéke, az egyed annál megfelelőbb.
    \item \textbf{szelekció:} rátermettség-arányos választással, súlyozott random generátornak köszönhetően. Miután a populáció egyedeit a célfüggvény-értékek alapján sorrendbe rakta a rendező metódusunk, a súlyozott random generátor annak megfelelő valószínűséggel szelektálja ki szülőnek az egyedeket, amilyen helyezést foglalnak el a listán. Ahány egyedből áll a populáció, a legmegfelelőbb egyednek annyiszor nagyobb esélye lesz, mint a leggyengébb egyednek. Amire még oda kell figyelni, hogy ugyanaz az egyed ne kerülhessen kétszer is kiválasztásra a szülők meghatározásakor, mivel az azt jelentené hogy a gyerek anyja egyben az apja is (igaz, a South Park rajzfilmsorozatban találkozhatunk ilyennel)...
    \item \textbf{keresztezés:} egypontos keresztezéssel. Véletlenszám-generálással döntjük el, hol legyen a keresztezési pont. A nehézséget annak megoldása jelenti, hogy ugyanaz az időablak (vagyis ugyanaz a gén) ne fordulhasson elő többször a gyermek egyedben. A keresztezési pont után jelentkezik ez a probléma és úgy küszöböljük ki, hogy amennyiben már megtalálható a gyerek génjei között a szülő adott génje, akkor a szülő keresztezési pont előtti génjei között megkeressük az első olyat, amely még nem.
    \item \textbf{mutáció:} ha mutálnunk kell a gyereket, először is véletlenszerűen generáljuk, hogy melyik legyen a mutálandó gén, és szintén véletlenszerűen generálunk egy idegen egyedet (vagyis olyan egyedet, amely nem szülője a gyereknek). Az idegen egyed sorrendileg utolsó génjének megfelelő gént megkeressük a gyerek génjei között, és azt felcseréljük a mutálandó génnel.
\end{itemize}

\noindent \textbf{A feladat formalizálása}

\begin{itemize}
    \item P: populáció mérete
    \item G: generációk száma
    \item $n \in \Bbb{N}$: osztály-tantárgy kettősök száma
    \item $o \in \Bbb{N}^n$: osztály-tantárgy kettősök heti óraszámai
    \item $m \in \Bbb{N}$: tanárok száma
    \item $r \in \Bbb{N}$: időablakok száma
    \item $b \in \{1;5\}$: tanárok \textsl{balance} értékei
    \item $e \in \{1;5\}$: tanárok \textsl{extremisms} értékei
    \item $c: \Bbb{N} \rightarrow \Bbb{N}$: célfüggvény
\end{itemize}

$$ c=\sum_{j=1}^m \sum_{l=1}^r \quad
\begin{cases}
b_j, & \hbox{ha a \textsl{balance}-szal kapcsolatos feltétel teljesül }l\hbox{ időablakban,}\\
e_j, & \hbox{ha a \textsl{extremisms}-zel kapcsolatos feltétel teljesül }l\hbox{ időablakban,}\\
b_j+e_j, & \hbox{ha a \textsl{balance}-szal és az \textsl{extremisms}-zel kapcsolatos feltétel is teljesül }l\hbox{ időablakban,}\\
0 & \hbox{egyébként.}
\end{cases} 
\rightarrow min$$

Összes lehetséges esetek száma:

$$P=r^r$$
Összes lehetséges megoldások száma:

$$P=r!$$

Célfüggvény növekedési rendje: $T(m,n,o)=\Theta(mno)$.

Keresztezés növekedési rendje: $T(I)=\sim \Theta(I^2)$.

Mutáció növekedési rendje: $T(I)=\sim \Theta(I)$.

Növekedési rend: $T(P,G,I,n,m,o)=\sim \Theta(G \cdot P \cdot \Theta(nmo) \cdot \Theta(I^2) \cdot \Theta(I))$.

$$
\begin{tabular}{|l|c|c|}
\hline
\multicolumn{3}{|c|}{Gyerek egyed}\\
\hline
& \bf{Nap} & \bf{Periódus}\\
\hline
Időablak\_1 & hétfő & 17-18\\
\hline
Időablak\_2 & csütörtök & 18-19\\
\hline
Időablak\_3 & hétfő & 17-18\\
\hline
Időablak\_4 & hétfő & 18-19\\
\hline
\end{tabular}
$$

\bibliographystyle{acm}
\bibliography{references}
\noindent [1]\quad $https://www.uni-miskolc.hu/~matka/Dokumentumok/EP\_feladatok.pdf$
 
\noindent [2]\quad D. Abramson, J. Abela: A parallel genetic algorithm for solving the school timetabling problem.
Commonwealth Scientific and Industrial Scientific Organisation (CSIRO), Australia, 1991.

\noindent [3]\quad F. G. Lobo, D. E. Goldberg, M. Pelikan: Time complexity of genetic algorithms on exponentially scaled problems.
University of Illinois, USA, 2000. 


\end{document}