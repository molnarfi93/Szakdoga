\documentclass[a4paper,12pt]{article}

% Set margins
\usepackage[hmargin=2.5cm, vmargin=3cm]{geometry}

\frenchspacing

% Language packages
\usepackage[utf8]{inputenc}
\usepackage[T1]{fontenc}
\usepackage[magyar]{babel}

% AMS
\usepackage{amssymb,amsmath}

% Graphic packages
\usepackage{graphicx}

% Colors
\usepackage{color}
\usepackage[usenames,dvipsnames]{xcolor}

% Enumeration
\usepackage{enumitem}

% Links
\usepackage{hyperref}

% Question
\newenvironment{question}[1]
{\noindent\textcolor{OliveGreen}{$\circ$ \textit{#1}}

\smallskip

\color{Gray}

}{\bigskip}

% Task
\newenvironment{task}[1]
{\noindent\textcolor{RoyalBlue}{$\circ$ \textit{#1}}

\smallskip

\color{Gray}

}{\bigskip}

% Notification
\newenvironment{notification}[1]
{\noindent\textcolor{Peach}{$\circ$ \textit{#1}}

\smallskip

\color{Gray}

}{\bigskip}

% Problem
\newenvironment{problem}[1]
{\noindent\textcolor{OrangeRed}{$\circ$ \textit{#1}}

\smallskip

\color{Gray}

}{\bigskip}

% Solution
\newenvironment{solution}
{\noindent\color{Violet}}{\bigskip}

% Starred
\newenvironment{starred}
{\noindent\color{Maroon}}{\bigskip}

\begin{document}

\begin{center}
    \huge \textbf{Optimalizálási problémák} 
\end{center}

\section{Kéttényezős változat}

Mielőtt nekilátnék a nagy műnek, a genetikus algoritmus létrehozásának, még további
információkat kell szerezni a probléma lemodellezésének teljessé tételéhez. Ezt úgy fogom
megtenni, hogy tényezők elhagyásával egyszerűsített problémákat oldok meg hagyományos
optimalizálási módszerek használatával. Először egy kéttényezős problémát hozok létre azzal,
hogy elhagyom a tanárokat és tantárgyakat, így csak az osztályokat kell hozzárendelni
optimálisan a tantermekhez. Így kapunk egy úgynevezett halmazfelbontási feladatot, ami arról
szól hogy az egyik halmaz (osztályok) és másik halmaz (tantermek) elemeihez 1:1
hozzárendelést kell elvégezni, vagyis minden osztályhoz pontosan 1 tanteremnek kell tartoznia
és minden tanteremhez pontosan 1 osztálynak.
Mindezt úgy hogy a legoptimálisabb megoldást
kapjuk a kihasználatlanság minimalizálásának szempontjából.

A kihasználatlanság az adott
terem kapacitásának és az adott osztály létszámának a különbsége.

Feltétel nyilván, hogy egy
osztálynak nem lehet órája olyan teremben, amelynek a kapacitása kisebb az osztály
létszámánál, a másik pedig, hogy egy évfolyam osztályai és az évfolyamon képzett nyelvi/
fakultációs csoportok nem lehetnek benne egyazon megoldásban, vagyis egy adott időablakban.

Mivel ebben az esetben könnyen előfordulhatna, hogy egyes diákoknak két helyen lenne jelenése
egyidőben. Mint látható lesz, ezt úgy oldottam meg, hogy minden egyes nyelvi és fakultációs
csoportot önálló osztályként kezeltem és a különböző évfolyamok nyelvi/fakultációs óráinak
számításba vétele esetén különböző hozzárendeléseket végeztem. Így összesen 7 hozzárendelést
kaptam, egyet arra az esetre, amikor kizárólag "alaposztályok" vannak, mivel mind a négy
évfolyamon vannak nyelvi órák, így összesen négyet a nyelvi órákat is tartalmazó
időablakokra és kettőt a fakultációs órákat is tartalmazó időablakokra, mert faktos órák
csak a 11. és 12. évfolyamon vannak. 

\subsection{A feladat formalizálása}

Jelölések:
\begin{itemize}
    \item $n \in \Bbb{N}$: osztályok száma
    \item $m \in \Bbb{N}$: tantermek száma
    \item $l \in \Bbb{N}^n$: osztályok létszámai
    \item $k \in \Bbb{N}^m$: tantermek kapacitásai
    \item $C \in \Bbb{N}^{n \times m}$: költségmátrix
    \item $A \in \{0;1\}^{n \times m}$: párosításmátrix
    \item $X \in \{0;1\}^{n \times m}$: hozzárendelés-mátrix
\end{itemize}

\[
C_{ij} =
\begin{cases}
k_j-l_i, & \hbox{ha } X_{ij}=1, \\
\infty & \hbox{egyébként}.
\end{cases}
\]

\[
A_{ij} =
\begin{cases}
1, & \hbox{ha } l_i \leq k_j, \\
0, & \hbox{egyébként}.
\end{cases}
\]

\[
X_{ij} =
\begin{cases}
1, & \hbox{ha az $i$-edik osztálynak a $j$-edik teremben lesz a tanóra}, \\
0&
\hbox{egyébként}.
\end{cases}
\]

$$\sum_{j=1}^m C_jX_j \rightarrow \hbox{min}$$

$$\sum_{j=1}^m A_{ij}X_j=1$$

$$i=1, 2, \ldots, n$$

$$j=1, 2, \ldots, m$$
Összes lehetséges esetek száma:

$$
P=\prod_{i=1}^n h_i \qquad \hbox{, ahol } h_i \hbox{ azon termek száma, ahová az } i \hbox{-edik osztály befér.}
$$
Összes lehetséges megoldások száma:

$$
P=\prod_{i=1}^n h_i-(i-1) \qquad \hbox{, ahol } h_i \hbox{ azon termek száma, ahová az } i \hbox{-edik osztály befér.}
$$  
A kapott optimumok:
\begin{itemize}
    \item 12 alapesetben
    \item 25 abban az esetben, amikor a 9. évfolyamon nyelvi órák vannak
    \item 14 abban az esetben, amikor a 10. évfolyamon nyelvi órák vannak
    \item 24 abban az esetben, amikor a 11. évfolyamon nyelvi órák vannak
    \item 23 abban az esetben, amikor a 12. évfolyamon nyelvi órák vannak
    \item 173 abban az esetben, amikor a 11. évfolyamon fakultációs órák vannak
    \item 171 abban az esetben, amikor a 12. évfolyamon fakultációs órák vannak
\end{itemize}

Nézzük az összes lehetséges esetek és lehetséges megoldások számát az alapesetben, az eredeti példa esetén,
illetve évfolyamonként 6, 7, 8 osztályra kibővített esetben (ahol a termek számát plusz egy évfolyamonkénti osztály
esetén 5-tel növeltem): 

$$
\begin{tabular}{|l|c|c|c|c|}
\hline
& 5 & 6 & 7 & 8 \\
\hline
Összes esetek & $1,1 \cdot 10^{23}$ & $1,4 \cdot 10^{31}$ & $1,7 \cdot 10^{38}$ & $2 \cdot 10^{46}$ \\ 
\hline
Összes megoldások & $5,3 \cdot 10^{16}$ & $3,5 \cdot 10^{22}$ & $1,5 \cdot 10^{26}$ & $3,3 \cdot 10^{31}$ \\
\hline
\end{tabular}
$$
A növekedési rend a következő: $\Theta (2nm).$

\section{Háromtényezős változat}

A háromtényezős esetben osztályokat, tanárokat és tantermeket rendeltem egymáshoz. Ehhez a
halmazlefedési feladatot alkalmaztam, ami abban különbözik a halmazfelbontásitól, hogy
1:N hozzárendelés van, ugyanis egy tanárhoz több osztályt és tantárgyat is rendelhetünk, sőt
ugye kell is többet hozzárendelni. Mivel a hagyományos optimalizálási módszerek esetén két
tényezővel tudunk dolgozni (ezért is lesz hasznos a mesterséges intelligencia nyújtotta
optimalizálási módszer, a genetikus algoritmus használata), így a háromból két tényezőt
összevontam. A tanárok és tantárgyak kettőse így 1 entitást alkot, ezáltal egy adott osztályt
annyiszor hoztam létre, ahány tantárgy van az adott évfolyamon. A párosításmátrixban két
feltételtől is függ, hogy 0 vagy 1 kerül a rublikába. Egyrészt, hogy az adott tanár tudja-e
tanítani az adott tantárgyat, illetve hogy az adott osztálynak van-e ilyen tárgya (a 11. és
12. évfolyamon pl. nincs fizika, csak fakt van belőle). A minimalizálás pedig itt arra 
vonatkozik, hogy minél kisebb legyen a tanárok heti óraszámai közötti eltérés, ne fordulhasson
elő, hogy mondjuk míg valaki 30 órát tart egy héten, addig más 5-öt. A futtatás után kapott
eredményt látva megállapítható, hogy amennyire lehetett, sikerült kiküszöbölni a tanárok
egyenlőtlen terhelését, megkaptuk a lehető legoptimálisabb osztály-tanár-tantárgy
hozzárendelés-mátrixot, amely meghatározza, hogy egy adott osztálynak egy adott tantárgyat
melyik tanár tartsa.

\subsection{A feladat formalizálása}

Jelölések:
\begin{itemize}
    \item $n$: osztály-tantárgy kettősök száma
    \item $m$: tanárok száma
    \item $t \in \Bbb{N}^m$: tanárok heti óraszámai
    \item $o \in \Bbb{N}^n$: osztály-tantárgy kettősökhöz tartozó heti óraszámok
    \item $A \in \{0;1\}^{n \times m}$: párosításmátrix
    \item $X \in \{0;1\}^{n \times m}$: hozzárendelés-mátrix
\end{itemize}

\[
t_{j} =
\begin{cases}
t_j+o_i,& \hbox{ha } X_{ij}=1, \\
t_j, & \hbox{egyébként}.
\end{cases}
\]

\[
A_{ij} =
\begin{cases}
1, & \hbox{ha az $i$-edik osztály-tantárgy kettősben szereplő tantárgyat tudja tanítani a $j$-edik tanár} \\
0, & \hbox{egyébként}.
\end{cases}
\]

\[
X_{ij} =
\begin{cases}
1, & \hbox{ha az $i$-edik osztály-tantárgy kettősben szereplő osztálynak az ugyanezen kettősben szereplő tantárgyat a $j$-edik tanár fogja tartani} \\
0, & \hbox{egyébként}.
\end{cases}
\]

$$\sum_{j=1}^m \vert o_jX_j-\overline{o}\vert \rightarrow \hbox{min}$$

$$\sum_{j=1}^m A_{ij} X_j=1$$

$$i=1, 2, \ldots, n$$

$$j=1, 2, \ldots, m$$
Összes lehetséges esetek száma:

$$P=m^n$$
Összes lehetséges megoldások száma:


$$P=\prod_{k=1}^p v_k^{u_k} \qquad \hbox{, ahol } p \hbox{ a különböző tantárgyak száma, } v_k \hbox{ megmutatja,
hogy a } k\hbox{-adik tárgyat hány különböző tanár tudja oktatni, az } u_k \hbox{ pedig hogy hány osztálynak és/vagy csoportnak kell tanulnia ezt a tárgyat}$$
A kapott optimum: 110.

A lehetséges esetek és megoldások száma (a tanárok számát plusz egy évfolyamonkénti osztály esetén 3-mal növeltem):

$$
\begin{tabular}{|l|c|c|c|c|}
\hline
& 5 & 6 & 7 & 8 \\
\hline
Összes esetek & $9,3 \cdot 10^{325}$ & $6,5 \cdot 10^{395}$ & $7,8 \cdot 10^{467}$ & $9,2 \cdot 10^{541}$ \\
\hline 
Összes megoldások & $1,1 \cdot 10^{132}$ & $2 \cdot 10^{168}$ & $4,2 \cdot 10^{208}$ & $1,2 \cdot 10^{249}$ \\
\hline
\end{tabular}
$$
Növekedési rend: $\Theta (nm+nm^2).$

\bibliographystyle{acm}
\bibliography{references}

\end{document}
