\documentclass[a4paper,12pt]{article}

% Set margins
\usepackage[hmargin=2.5cm, vmargin=3cm]{geometry}

\frenchspacing

% Language packages
\usepackage[utf8]{inputenc}
\usepackage[T1]{fontenc}
\usepackage[magyar]{babel}

% AMS
\usepackage{amssymb,amsmath}

% Graphic packages
\usepackage{graphicx}

% Colors
\usepackage{color}
\usepackage[usenames,dvipsnames]{xcolor}

% Enumeration
\usepackage{enumitem}

% Links
\usepackage{hyperref}

% Question
\newenvironment{question}[1]
{\noindent\textcolor{OliveGreen}{$\circ$ \textit{#1}}

\smallskip

\color{Gray}

}{\bigskip}

% Task
\newenvironment{task}[1]
{\noindent\textcolor{RoyalBlue}{$\circ$ \textit{#1}}

\smallskip

\color{Gray}

}{\bigskip}

% Notification
\newenvironment{notification}[1]
{\noindent\textcolor{Peach}{$\circ$ \textit{#1}}

\smallskip

\color{Gray}

}{\bigskip}

% Problem
\newenvironment{problem}[1]
{\noindent\textcolor{OrangeRed}{$\circ$ \textit{#1}}

\smallskip

\color{Gray}

}{\bigskip}

% Solution
\newenvironment{solution}
{\noindent\color{Violet}}{\bigskip}

% Starred
\newenvironment{starred}
{\noindent\color{Maroon}}{\bigskip}

\begin{document}

\begin{center}
    \huge \textbf{Órarendgenerálási probléma megoldási lehetőségei} 
\end{center}

\section{Kéttényezős feladat}

A genetikus algoritmus létrehozása előtt mindenképpen szükség van az eredeti  négytényezős 
probléma felbontására, hogy egyszerűbb, kéttényezős és háromtényezős problámákat 
kapjunk, melyek megoldhatóak hagyományos optimalizálási módszerekkel és együttesen 
alkotják a négytényezős optimalizálási feldat megoldásait. Az ezek során kapott eredmények
támpontot adnak a genetikus algoritmus kapcsán és megállapíthatjuk, hogy egyáltalán
érdemes-e genetikus algoritmust használnunk ezen probléma megoldására. Először egy 
kéttényezős problémát hozok létre azzal, hogy elhagyom a tanárokat és tantárgyakat, így csak
az osztályokat kell hozzárendelni optimálisan a tantermekhez. Így kapunk egy úgynevezett
halmazfelbontási feladatot, ami arról szól hogy az egyik halmaz (osztályok) és másik halmaz 
(tantermek) elemeihez 1:1 hozzárendelést kell elvégezni, vagyis minden osztályhoz pontosan 1
tanteremnek kell tartoznia és minden tanteremhez pontosan 1 osztálynak. Mindezt úgy hogy a
legoptimálisabb megoldást kapjuk a kihasználatlanság minimalizálásának szempontjából. A
kihasználatlanság az adott terem kapacitásának és az adott osztály létszámának a különbsége.
Feltétel nyilván, hogy egy osztálynak nem lehet órája olyan teremben, amelynek a kapacitása
kisebb az osztály létszámánál, a másik pedig, hogy egy évfolyam osztályai és az évfolyamon 
képzett nyelvi/fakultációs csoportok nem lehetnek benne egyazon megoldásban, vagyis egy
adott időablakban. Mivel ebben az esetben könnyen előfordulhatna, hogy egyes diákoknak 
két helyen lenne jelenése egyidőben. Mint látható lesz, ezt úgy oldottam meg, hogy minden
egyes nyelvi és fakultációs csoportot önálló osztályként kezeltem és a különböző évfolyamok
nyelvi/fakultációs óráinak számításba vétele esetén különböző hozzárendeléseket végeztem. 
Így összesen 7 hozzárendelést kaptam, egyet arra az esetre, amikor kizárólag "alaposztályok" 
vannak, mivel mind a négy évfolyamon vannak nyelvi órák, így összesen négyet a nyelvi 
órákat is tartalmazó időablakokra és kettőt a fakultációs órákat is tartalmazó időablakokra, 
mert faktos órák csak a 11. és 12. évfolyamon vannak. 

\noindent \textbf{A feladat formalizálása}

\begin{itemize}
    \item $n \in \Bbb{N}$: osztályok száma
    \item $m \in \Bbb{N}$: tantermek száma
    \item $l \in \Bbb{N}^n$: osztályok létszáma
    \item $k \in \Bbb{N}^m$: tantermek kapacitása
    \item $h \in \Bbb{N}^n$: termek száma, ahová az egyes osztályok beférnek
    \item $C \in \Bbb{N}^{n \times m}$: költségmátrix
    \item $A \in \{0;1\}^{n \times m}$: párosításmátrix
    \item $X \in \{0;1\}^{n \times m}$: hozzárendelés-mátrix
\end{itemize}

\[
C_{ij} =
\begin{cases}
k_j-l_i, & \hbox{ha } X_{ij}=1, \\
\infty & \hbox{egyébként}.
\end{cases}
\]

\[
A_{ij} =
\begin{cases}
1, & \hbox{ha } l_i \leq k_j, \\
0, & \hbox{egyébként}.
\end{cases}
\]

\[
X_{ij} =
\begin{cases}
1, & \hbox{ha az $i$-edik osztálynak a $j$-edik teremben lesz a tanóra}, \\
0&
\hbox{egyébként}.
\end{cases}
\]

$$\sum_{j=1}^m C_jX_j \rightarrow \hbox{min}$$

$$\sum_{j=1}^m A_{ij}X_j=1$$

$$i=1, 2, \ldots, n$$

$$j=1, 2, \ldots, m$$
Összes lehetséges esetek száma:

$$P=\prod_{i=1}^n h_i$$
Összes lehetséges megoldások száma:

$$P=\prod_{i=1}^n h_i-(i-1)$$
Nézzük az összes lehetséges esetek és lehetséges megoldások számát az alapesetben, az eredeti példa esetén,
illetve évfolyamonként 6, 7, 8 osztályra kibővített esetben (ahol a termek számát plusz egy évfolyamonkénti osztály
esetén 5-tel növeltem): 

$$
\begin{tabular}{|l|c|c|c|c|}
\hline
& 5 & 6 & 7 & 8 \\
\hline
Összes esetek & $1,1 \cdot 10^{23}$ & $1,4 \cdot 10^{31}$ & $1,7 \cdot 10^{38}$ & $2 \cdot 10^{46}$ \\ 
\hline
Összes megoldások & $5,3 \cdot 10^{16}$ & $3,5 \cdot 10^{22}$ & $1,5 \cdot 10^{26}$ & $3,3 \cdot 10^{31}$ \\
\hline
\end{tabular}
$$
A növekedési rend a következő: $T(n,m)=\sim \Theta (2nm).$

$$
\begin{tabular}{|c|c|c|c|}
\hline
5 & 6 & 7 & 8 \\
\hline
$3600$ & $4760$ & $6080$ & $7560$ \\ 
\hline
\end{tabular}
$$

\section{Háromtényezős feladat}

A háromtényezős esetben osztályokat, tanárokat és tantárgyakat rendeltem egymáshoz. Ehhez 
a halmazlefedési feladatot alkalmaztam, ami abban különbözik a halmazfelbontásitól, hogy 1:N
hozzárendelés van, ugyanis egy tanárhoz több osztályt és tantárgyat is rendelhetünk, sőt ugye
kell is többet hozzárendelni. Mivel a hagyományos optimalizálási módszerek esetén két tényezővel
tudunk dolgozni (ezért is lesz hasznos a mesterséges intelligencia nyújtotta optimalizálási módszer,
a genetikus algoritmus használata), így a háromból két tényezőt összevontam. A tanárok és 
tantárgyak kettőse így 1 entitást alkot, ezáltal egy adott osztályt annyiszor hoztam létre, ahány 
tantárgy van az adott évfolyamon. A párosításmátrixban két feltételtől is függ, hogy 0 vagy 1 
kerül a rublikába. Egyrészt, hogy az adott tanár tudja-e tanítani az adott tantárgyat, illetve hogy
az adott osztálynak van-e ilyen tárgya (a 11. és 12. évfolyamon pl. nincs fizika, csak fakt van 
belőle). A minimalizálás pedig itt arra vonatkozik, hogy minél kisebb legyen a tanárok heti 
óraszámai közötti eltérés, ne fordulhasson elő, hogy mondjuk míg valaki 30 órát tart egy héten,
addig más 5-öt. A futtatás után kapott eredményt látva megállapítható, hogy amennyire lehetett,
sikerült kiküszöbölni a tanárok egyenlőtlen terhelését, megkaptuk a lehető legoptimálisabb osztály-
tanár-tantárgy hozzárendelés-mátrixot, amely meghatározza, hogy egy adott osztálynak egy
adott tantárgyat melyik tanár tartsa.

\noindent \textbf{A feladat formalizálása}

\begin{itemize}
    \item $n$: osztály-tantárgy kettősök száma
    \item $m$: tanárok száma
    \item $p$: tantárgyak száma
    \item $u \in \Bbb{N}^p$: az egyes tárgyakat hallgató osztályok száma
    \item $v \in \Bbb{N}^p$: az egyes tárgyakat oktató tanárok száma
    \item $t \in \Bbb{N}^m$: tanárok heti óraszáma
    \item $o \in \Bbb{N}^n$: osztály-tantárgy kettősök heti óraszáma
    \item $A \in \{0;1\}^{n \times m}$: párosításmátrix
    \item $X \in \{0;1\}^{n \times m}$: hozzárendelés-mátrix
\end{itemize}

\[
t_{j} =
\begin{cases}
t_j+o_i,& \hbox{ha } X_{ij}=1, \\
t_j, & \hbox{egyébként}.
\end{cases}
\]

\[
A_{ij} =
\begin{cases}
1, & \hbox{ha az $i$-edik osztály-tantárgy kettősben szereplő tantárgyat tudja tanítani a $j$-edik tanár} \\
0, & \hbox{egyébként}.
\end{cases}
\]

\[
X_{ij} =
\begin{cases}
1, & \hbox{ha az $i$-edik osztály-tantárgy kettősben szereplő osztálynak az ugyanezen kettősben szereplő tantárgyat a $j$-edik tanár fogja tartani} \\
0, & \hbox{egyébként}.
\end{cases}
\]

$$\sum_{j=1}^m \vert o_jX_j-\overline{o}\vert \rightarrow \hbox{min}$$

$$\sum_{j=1}^m A_{ij} X_j=1$$

$$k=1, 2, \ldots, p$$
Összes lehetséges esetek száma:

$$P=m^n$$
Összes lehetséges megoldások száma:

$$P=\prod_{k=1}^p v_k^{u_k}$$
A lehetséges esetek és megoldások száma (a tanárok számát plusz egy évfolyamonkénti osztály esetén 3-mal növeltem):

$$
\begin{tabular}{|l|c|c|c|c|}
\hline
& 5 & 6 & 7 & 8 \\
\hline
Összes esetek & $9,3 \cdot 10^{325}$ & $6,5 \cdot 10^{395}$ & $7,8 \cdot 10^{467}$ & $9,2 \cdot 10^{541}$ \\
\hline 
Összes megoldások & $1,1 \cdot 10^{132}$ & $2 \cdot 10^{168}$ & $4,2 \cdot 10^{208}$ & $1,2 \cdot 10^{249}$ \\
\hline
\end{tabular}
$$
Növekedési rend: $T(n,m)=\Theta (nm+nm^2).$

$$
\begin{tabular}{|c|c|c|c|}
\hline
5 & 6 & 7 & 8 \\
\hline
$19800$ & $25740$ & $32400$ & $39780$ \\
\hline
\end{tabular}
$$

\section{Négytényezős feladat}

A négytényezős feladat megoldása a két- és háromtényezős feladat megoldásainak egyesítéséből áll össze, vagypedig
genetikus algoritmus megírása által. Akármelyiket is választjuk, az összes és lehetséges megoldások száma a háromtényezős
feladatnál kapott szám lesz, mivel a kéttényezős feladat eset- és megoldásszámai nagyságrendileg elhanyagolhatóak
ezekhez képest. A különbség a növekedési rendben lesz, ami a két- és háromtényezős feladat megoldásainak egyesítése
esetén a két növekedési rend összegeként áll elő.

\noindent A probléma leképezése a genetikus algoritmus összetevőire:
\begin{itemize}
    \item \textbf{egyed:} időablak (pl. kedd 10-11 óra), melyben tanórák kerülnek megtartásra, különböző
        termekben, különböző tanárokkal, különböző osztályoknak. Az időintervallum nem kerül
        rögzítésre az egyes egyedeknél, mert ez esetben nem lehetne genetikus algoritmussal
        dolgozni egyrészt, másrészt szükség sincs rá, mert sorrendileg bármelyik időablak
        felcserélhető lesz bármelyik időablakkal, miután elértük a célt: az
        ütközésmentességet. Python nyelvet használok, ahol az implementáció szótár típus lesz,
        3 adattaggal: kulcs szerepű azonosító (egész szám), az egyedhez aktuálisan tartozó
        gének, vagyis az időablakban levő tanórákra való referenciák (tömb), az optimumtól
        való eltérés, vagyis a büntetőfüggvények értékei (tömb).
    \item \textbf{populáció:} az egyedek összessége. Ez a heti összes időablakok száma lesz. Esetünkben hétfőtől
        szerdáig napi 8 időablakot állapítok meg, csütörtökre és péntekre napi 6-ot. Így lesz 36 fős populációnk,
        melyet listában tárolunk.  
    \item \textbf{gén:} tanóra, ami egy osztály-tanár-tanterem-tantárgy négyes, plusz egy azonosító, melynek
      segítségével lehet hivatkozni a tanórára. Implementációja szótár. A gének összességét
      pedig listában tároljuk. Egy adott tantárgy egy időablakban többször is előfordulhat,
      viszont egy osztály, tanár, tanterem csak egyszer, ezért ezeknek a kapcsán
      büntetőfüggvényt tartunk számon. Ha többször is előfordul az adott időablakban mondjuk
      egy osztály, akkor az osztály-büntetőfüggvény értékét növeljük az előfordulások 
      száma-1 értékkel, miután az összes osztályt végigvettük, kialakul a büntetőfüggvény
      végső értéke. Ugyanígy járunk el a tanárok és tantermek esetében is. Ezekhez az
      értékekhez a tanórák kulcs adattagján keresztül fér hozzá az időablak egyed.
    \item \textbf{célfüggvény:} a 3-féle büntetőfüggvény összesítésével kapott függvény, melynek optimális
              értéke 0. Csak ebben az esetben nincs ütközés. A büntetőfüggvények értékeit
              súlyozva vesszük számításba annak megfelelően, hogy az adathalmazban mik
              a darabszámok egymáshoz viszonyított arányai. Az én adathalmazomban 30 tanterem,
              30 tanár és 60 osztály lesz, de utóbbira még külön ki kell térni. A 60 osztály
              között lesz 20 "alaposztály", 20 nyelvi és 20 fakultációs csoport. Ezáltal a tanterem- és
              tanárbüntetőfüggvény értéke 1,5-ös súllyal fog szorzódni az osztály-büntetőfüggvényhez
              képest. Miután összeadjuk őket, megkapjuk a célfüggvény értékét.
    \item \textbf{öröklődés:} egypontos keresztezéssel. A kromoszómák hossza, vagyis az egyedek génjeinek
            száma eltérő lehet, ezért maximálnunk kell, hogy az 1. és hanyadik gén között
            lehessen keresztezési pont, vagyis mi legyen a véletlenszám-generálás felső
            határa. Ezt a maximális értéket az adathalmaz méretének ismeretében határozzuk
            meg, és ez így egyben a keresztezés valószínűségi értékének meghatározására is
            alkalmas lehet.
    \item \textbf{mutáció:} ha egy új egyed valamely génjét mutálnunk kell, úgy oldjuk meg, hogy egy
          véletlenszerűen választott idegen egyed (vagyis nem szülő) sorrendileg utolsó
          génjét töröljük az idegen egyedből és ugyanakkor az egyedünk mutált génjévé
          tesszük.
    \item \textbf{szelekció:} rátermettség-arányos választással.
\end{itemize}

\noindent \textbf{A feladat formalizálása}

\begin{itemize}
    \item $n \in \Bbb{N}$: osztály-tantárgy kettősök száma
    \item $m \in \Bbb{N}$: tanárok száma
    \item $p \in \Bbb{N}$: tantermek száma
    \item $c \in \Bbb{N}^n$: osztály-tantárgy kettősök előfordulásainak száma
    \item $e \in \Bbb{N}^m$: tanárok előfordulásainak száma
    \item $h \in \Bbb{N}^p$: tantermek előfordulásainak száma
    \item $o \in \Bbb{N}^n$: osztály-tanár kettősök heti óraszáma
    \item $I \in \Bbb{N}$: időablakok száma
\end{itemize}

$$\sum_{i=1}^n c_i \in (0,1), \quad \hbox{minden időablak esetén.}$$

$$\sum_{j=1}^m e_j \in (0,1), \quad \hbox{minden időablak esetén.}$$

$$\sum_{k=1}^p h_k \in (0,1), \quad \hbox{minden időablak esetén.}$$

$$\sum_{i=1}^n c_i - o_i=0$$

Összes lehetséges esetek száma:

$$P=\binom{n+m-1}{m}$$
Összes lehetséges megoldások száma:

$$P=\binom{n}{m}$$
A lehetséges esetek és megoldások száma (a tanárok számát plusz egy évfolyamonkénti osztály esetén 3-mal növeltem):

$$
\begin{tabular}{|l|c|c|c|c|}
\hline
& 5 & 6 & 7 & 8 \\
\hline
Összes esetek & $5 \cdot 10^{105}$ & $9,1 \cdot 10^{119}$ & $1,7 \cdot 10^{134}$ & $3,5 \cdot 10^{148}$ \\
\hline 
Összes megoldások & $8,9 \cdot 10^{36}$ & $6,8 \cdot 10^{41}$ & $4,5 \cdot 10^{46}$ & $2,7 \cdot 10^{51}$ \\
\hline
\end{tabular}
$$
Növekedési rend: $T(n,o,I)=\sim \Theta(noI)$.

$$
\begin{tabular}{|c|c|c|c|}
\hline
5 & 6 & 7 & 8 \\
\hline
$24400$ & $28960$ & $33520$ & $38080$ \\
\hline
\end{tabular}
$$

\begin{itemize}
    \item P: populáció mérete
    \item G: generációk száma
    \item $n \in \Bbb{N}$: osztály-tantárgy kettősök száma
    \item $c \in \Bbb{N}^n$: osztály-tantárgy kettősök előfordulásainak száma
    \item $o \in \Bbb{N}^n$: osztály-tantárgy kettősök heti óraszáma
    \item $b: \Bbb{N} \rightarrow \Bbb{N}$: büntetőfüggvény
\end{itemize}

$$ b(n)=\sum_{i=1}^n \vert c_i-o_i\vert \rightarrow 0$$

Célfüggvény növekedési rendje: $T(n,I)=\Theta(nI)$.

Keresztezés növekedési rendje: $T(I)=\Theta(I)$.

Mutáció növekedési rendje: $T=\sim \Theta(0)$.

Növekedési rend: $T=\sim \Theta(G \cdot P \cdot \Theta(nI) \cdot \Theta(I))$.

$$
\begin{tabular}{|l|c|c|c|c|}
\hline
\multicolumn{5}{|c|}{Gyerek egyed}\\
\hline
Időablak\_1 & & & &\\
\hline
& \bf{Osztály} & \bf{Tantárgy} & \bf{Terem} & \bf{Tanár}\\
\hline
& 9. & matek & II. & Molnárfi\\
\hline 
& 10. & matek & IV. & Balogh\\
\hline
& 11. & infó & III. & Molnár\\
\hline
Időablak\_2 & & & &\\
\hline
& \bf{Osztály} & \bf{Tantárgy} & \bf{Terem} & \bf{Tanár}\\
\hline
& 9. & matek & II. & Szabó\\
\hline
& 10. & matek & IV. & Balogh\\
\hline
Időablak\_3 & & & &\\
\hline
& \bf{Osztály} & \bf{Tantárgy} & \bf{Terem} & \bf{Tanár}\\
\hline
& 9. & infó & II. & Molnár\\
\hline
& 10. & infó & IV. & Molnárfi\\
\hline
& 11. & matek & III. & Balogh\\
\hline
& 12. & matek & I. & Szabó\\
\hline
Időablak\_4 & & & &\\
\hline
& \bf{Osztály} & \bf{Tantárgy} & \bf{Terem} & \bf{Tanár}\\
\hline
& 9. & infó & II. & Molnár\\
\hline
& 10. & infó & IV. & Molnárfi\\
\hline
& 11. & matek & III. & Balogh\\
\hline
& 12. & matek & I. & Szabó\\
\hline
\end{tabular}
$$

$$
\begin{tabular}{|l|c|c|c|c|}
\hline
\multicolumn{5}{|c|}{Gyerek egyed}\\
\hline
Időablak\_1 & & & &\\
\hline
& \bf{Osztály} & \bf{Tantárgy} & \bf{Terem} & \bf{Tanár}\\
\hline
& 9. & matek & II. & Szabó\\
\hline 
& 10. & matek & IV. & Balogh\\
\hline
& 11. & infó & III. & Molnár\\
\hline
& 12. & infó & I. & Molnárfi\\
\hline
Időablak\_2 & & & &\\
\hline
& \bf{Osztály} & \bf{Tantárgy} & \bf{Terem} & \bf{Tanár}\\
\hline
& 9. & matek & II. & Molnárfi\\
\hline
& 10. & matek & IV. & Balogh\\
\hline
& 12. & infó & I. & Molnár\\
\hline
Időablak\_3 & & & &\\
\hline
& \bf{Osztály} & \bf{Tantárgy} & \bf{Terem} & \bf{Tanár}\\
\hline
& 9. & infó & II. & Balogh\\
\hline
& 10. & infó & IV. & Molnárfi\\
\hline
& 11. & matek & III. & Szabó\\
\hline
& 12. & matek & I. & Molnár\\
\hline
Időablak\_4 & & & &\\
\hline
& \bf{Osztály} & \bf{Tantárgy} & \bf{Terem} & \bf{Tanár}\\
\hline
& 9. & infó & II. & Balogh\\
\hline
& 10. & infó & IV. & Molnárfi\\
\hline
& 11. & matek & III. & Szabó\\
\hline
& 12. & matek & I. & Molnár\\
\hline
\end{tabular}
$$

\bibliographystyle{acm}
\bibliography{references}
\noindent [1]\quad $https://www.uni-miskolc.hu/~matka/Dokumentumok/EP\_feladatok.pdf$
 
\noindent [2]\quad D. Abramson, J. Abela: A parallel genetic algorithm for solving the school timetabling problem.
Commonwealth Scientific and Industrial Scientific Organisation (CSIRO), Australia, 1991.

\noindent [3]\quad F. G. Lobo, D. E. Goldberg, M. Pelikan: Time complexity of genetic algorithms on exponentially scaled problems.
University of Illinois, USA, 2000. 


\end{document}