\Chapter{Az órarendgenerálási probléma}

\Section{A genetikus algoritmus}

\SubSection{Biológiai háttér}

Charles Darwin, angol biológus evolúcióelmélete paradigmaváltást hozott magával a biológiában. 
Lényege a fajok fennmaradásának, túlélésének genetikai háttére a Föld ökoszisztémáiban 
(életközösségeiben). Összetevői a szelekció, az öröklődés és a mutáció. A természetes 
szelekció értelmében a gyenge, vagyis kedvezőtlenebb, kevésbé optimális tulajdonságokkal 
rendelkező egyedeknek el kell pusztulniuk annak érdekében, hogy az adott faj populációja 
életképesebb legyen a következő generációkban. A kedvezőtlen gének továbbörökítése ezt
hátráltatná és mivel az állatvilágban táplálkozási hierarchia van, továbbá az éghajlati 
körülményekhez való alkalmazkodás is szerepet játszik a túlélésben, ebben az esetben félő 
lenne a populáció kihalása. A nőstények ösztönösen a kedvezőbb tulajdonságokkal rendelkező 
hímeket választják párosodásra, vagypedig egymás elleni csatában lejátsszák a hímek, ki 
juthasson hozzá a nőstényhez. Ez lehet nem halálos vagy halálos játszma is. Szintén kegyetlen 
világ figyelhető meg pl. a fókák esetében, ahol a győztes hím az összes nősténnyel párosodik, 
a vesztes pedig eggyel sem. Többnyire az erősebb egyedek örökítik így tovább a génjeiket, a
gyengébbek fokozatosan elhullanak a populációból, amely így generációról generációra erősebb,
életképesebb lesz. A mutáció szerepe is fontos, melynek eredménye hogy néha-néha egy egyed
valamely génje egyik szülőtől sem származó lesz, hanem egészen más lesz, következményként
nagymértékben különbözni fog mindkét szülőtől valamely tulajdonság kapcsán. Ez a jelenség is 
jó a populáció számára, mert csökkenti a belterjesség mértékét, a korai generációk folyamán
bekövetkező mutációk segíthetnek az adott ökoszisztémában hasznos, korábban nem jellemző
tulajdonságok térnyerésében, a későbbi mutációk már hamar elhalnak, nem játszanak szerepet. [1]

\SubSection{Általánosságban a genetikus algoritmusról}

A genetikus algoritmusokat olyan diszkrét optimalizálási problémák megoldására alkalmazzuk,
amelyekre nem létezik általános eljárás, a probléma specifikussága, egyedisége következtében.
Túl sok az ismeretlen és a közöttük levő összefüggések is ismeretlenek, így a rendszernek 
"tanulnia kell" a probléma megoldása során, vagyis mesterséges intelligenciát kell 
használnunk. A genetikus algoritmus épp megfelel. Ahogyan az evolúció is lassú folyamat volt,
úgy az evolúciót modellező genetikus algoritmusok is lassúak, ahhoz képest hogy áramkörökről
beszélhetünk a háttérben, de nyilván nagyságrendekkel hamarabb kapunk eredményt, mintha emberi
intelligenciával dolgoznánk és amennyiben van megoldás, akkor azt egy megfelelően elkészített
genetikus algoritmus biztosan megtalálja. Mivel megelőzően nem lehetünk biztosak benne, hogy
egészen megfelelő paramétereket választottunk, az algoritmus többszöri lefuttatásával 
szerezhetünk bizonyosságot a megoldás optimális értékeit illetően. Ezenkívül van rá lehetőség,
hogy az algoritmus párhuzamosításával növeljük a gyorsaságot és csökkentsük a program méretét,
ezáltal a memóriaigényt.
Ahogyan az evolúció törvényei következtében kitermelődtek az egyes fajok optimális 
tulajdonságokkal rendelkező egyedekből álló populációi, a túlélés érdekében, úgy termeljük ki
egy adathalmaz optimális értékekkel rendelkező tagjait az evolúció mozzanatait (szelekció,
öröklődés, mutáció) a maga módján leutánzó genetikus algoritmussal, az adott probléma 
megoldása érdekében. Ahogyan más területeken, úgy itt is leképezzük a valós világot a
programozás nyelvére, nézzük az analógiákat általánosságban:

\begin{itemize}
	\item populáció $\rightarrow$ adathalmaz
	\item egyed $\rightarrow$ adathalmaz egy tagja
	\item gén $\rightarrow$ adathalmaz egy tagjának egy attribútuma
	\item túléléshez szükséges tulajdonságok $\rightarrow$ adathalmaz tagjainak attribútumai
	\item ezen tulajdonságok kedvező mivolta $\rightarrow$ az attribútumok optimális értéke
\end{itemize}

A szelekció, öröklődés és mutáció függvények segítségével valósul meg, de ezekkel
kapcsolatban többféle lehetőség létezik és az adott problémánk ismeretében választjuk ki,
hogy milyen fajta szelekciós, öröklődési, illetve mutációs eljárást alkalmazunk. Ezekre nem
térek ki, csak az én munkám során használt eljárásokat fogom bemutatni.
A genetikus algoritmus esetleges sikertelenségét a lokális optimumok problematikája okozza. 
Azt képzeljük el, hogy egy hegyvidék legalacsonyabb völgyét kell megtalálnunk, de miután a
genetikus algoritmus talál egy völgyet (vagyis egy lokális minimumot), nem tudja megmondani,
hogy a legmélyebbet (vagyis egyben a globális optimumot) találta-e meg. A programozó feladata
annak elérése, hogy a rendszer ne gondolja azt tévesen, a globális optimumot találta meg. 
Máskülönben megreked az algoritmus, amely nem szeretne a völgyből megindulni felfelé, hiszen 
ekkor távolodunk az optimális megoldástól. Csakhogy ez az optimális megoldás könnyen lehet,
hogy csak lokális optimum, ami nekünk nem elég. Ahhoz, hogy a globális optimumot megtaláljuk,
átmenetileg el kell távolodnunk a már viszonylag jó megoldástól. Nézzük, miként tudjuk
módosítani a paramétereket a lokális minimumon való megrekedés elkerülése érdekében:

\begin{itemize}
	\item nagyobb populációval
	\item a keresztezések és mutációk valószínűségének növelésével
	\item generációnként kevesebb egyed kiszelektálásával
	\item de az is előfordulhat, hogy hibát követtünk el a probléma lemodellezése során
	\item ezenkívül egyes, rendkívüli összetettségű problémák esetén lehetséges és érdemes                 több, egymással kapcsolatban álló populációt létrehozni
\end{itemize}
[2]

\Section{Kitekintés}

Több, erre a célre létrehozott szoftver készült már, ingyenes/fizetős, platformfüggetlen/
valamely platformra létrehozott, interaktív/automatikus. Az interaktív használat itt azt
jelenti, hogy mi adunk meg órarendet, melynek ütközésmentességét leteszteljük a programmal,
ami jelzésünkre adja, mi mivel ütközik és milyen lehetőségeink vannak a kiküszöbölésre. 
A legnépszerűbb applikációkat hasonlítom össze:

\begin{itemize}
	\item \textbf{TimeTabler:} A legrégebbi, mai napig nagyon népszerű órarendütemező, olyan extra funkciókkal,
	mint a részmunkaidős tanárok felvétele a rendszerbe, a lyukas órák számának minimalizálása és a választható
	tanóra-időtartamok (szimpla vagy dupla órák). Interaktív módon is használható. Többféle exportálási lehetőséget biztosít.
	Ingyenesen letölthető egy demo verzió nem megváltoztatható, demo adatokkal és 
	tutorial videóval. Előfizetés nincs, csak megvásárlás lehetséges. 
	Platformfüggetlen, telefonos verzióval viszont nem rendelkezik. [4]
	\item \textbf{Schedule Builder:} Ingyenes, webes alkalmazás. Csak az alapfunkciókkal rendelkezik,
	nincsenek extrák, illetve nem elérhető az automatikus használat sem. [5]
	Platformfüggetlen, többféle exportálási lehetőséggel.
	\item \textbf{Prime TimeTable:} Macintosh, Linux, Android és IOS rendszereken egyaránt elérhető, viszont
	Windowson nem. Rendelkezik minden olyan funkcióval, amivel a TimeTabler. 30 napos próbaverzióval vagy éves
	előfizetéssel lehet használni. [6]
	\item \textbf{aSc TimeTables:} A közelmúltban legjobbnak választott órarendgeneráló applikáció. Itt is elérhető interaktív és
	automatikus használat, ugyanazok az extra funkciók, többféle exportálási lehetőség, próbaverzió. Előfizetés nincs, csak
	megvásárlás lehetséges, használata Windowsra és Macintoshra korlátozódik. [7]
\end{itemize}

Mint látható, mindegyik fizetős applikáció rendelkezik automatikus és interaktív használati
lehetőséggel, valamint ugyanazokkal az extra funkciókkal. A nyomtatási és exportálási lehetőségekkel
pedig az ingyenes Schedule Builder is. 

\Section{Felhasznált technológiák}

\SubSection{Python}

Nagyrészt a Guido van Rossum, holland programozó által alkotott Python multiparadigmás 
(vagyis a procedurális, objektumorientált, funkcionális programozást egyaránt támogató)
programnyelvet, annak 3.8.2-es verzióját használtam. Jellemzői még az interpreteres 
feldolgozás (a futtatás közvetlenül a forráskódból történik) és a platformfüggetlenség 
(mindegyik széles körben elterjedt operációs rendszeren használható). Pythonra épülő
ismertebb alkalmazások a Plone tartalomkezelő rendszer, a Blender animációs modellező
és a Torrent fájlmegosztó. Python nyelvben a tömbök alapértelmezetten dinamikusak, vagyis
listák, valamint a szótár típusú változó is rendelkezésre áll ebben a nyelvben, így 
kényelmes volt az adatkezelés. Utóbbi adattípus abban különbözik a listától, hogy az 
elemei rendelkeznek azonosító névvel vagy kóddal, ennek megadásával lehet rájuk hivatkozni, 
nem tömbindexszel, ami különböző típusú elemek esetén hasznos. A szemétgyűjtés 
automatikusan történik, nincsen szükség memóriafelszabadításra és memóriafoglalásra az 
algoritmus során kiszelektálódó, illetve létrejövő egyedek kapcsán, bár mint kiderült,
erre nem volt szükségem. [8]

\SubSection{Python kiegészítők}

Több kiegészítőt is telepítenem és használnom kellett a szerver létrehozásával kapcsolatban Python-ban, ezek a Falcon, a Waitress, a JWT és az SQLAlchemy. A Falcon egy olyan webes keretrendszer, amely a HTTP kérések (get, post, put, delete) és a REST (internetes szoftverarchitektúra) közötti  kapcsolatért felel. Vagyis egy interfész, amire azért van szükség, mert a REST nem korlátozódik a HTTP protokollra. Azért esett rá a választás, mert kifejezetten REST-hez készült és gyors. A Waitress többféle felhasználásra alkalmas, interfészt biztosít webszerver és alkalmazásszerver között. Arra használtam, hogy az adatbázissal kapcsolatos betöltési kérelmek kapcsán biztosítsa az elérést, host és port, valamint egy olyan objektum segítségével, amivel hozzáférünk a HTTP kérések eredményeihez. Azért ezt használtam, mert közvetlenül elindítható a Python kódból. A JWT a JSON Web Token rövidítése. Kizárólagos vezérlést biztosít a token, melyet a felhasználó egyedi azonosítójából (user hozzáadáskor automatikusan hozárendelt sorszám) és a programozó által megadott kulcsból generál a szerver, hash algoritmussal. Ezzel a titkosítással érjük el a jelszavak védelmét. Elterjedt, jól támogatott megoldás, úgy tárolja a token által az adatokat, hogy a kliens ne tudjon hozzáférni. Végül a Jinja, melyet a konzulensem hatékonyan használt a korábbiakban, a Python fájlkezelésében használt eszköz. Általa a html hozzájut a Python változóiban levő adatokhoz. [8]

\SubSection{SQL és SQLAlchemy}

Donald D. Chamberlin SQL nevű adatbáziskezelő nyelve, illetve az Oracle MySQL 
adatbáziskezelő szervere az, amelyet az adatok letárolására és kezelésére (lekérdezésére, 
módosítására) a háttérben használtam, a webfejlesztésben érdekelt komolyabb vállalatok 
többségéhez hasonlóan. Az SQL parancsnyelv relációs adatbázisok kezelésére lett létrehozva, 
az objektum-relációs leképezéshez pedig Python esetén SQLAlchemy-t használunk, így én is ezt 
tettem. Erre azért van szükség, mert ha objektum-orientált paradigmát követünk, az adatbázis 
relációs struktúrája ugyan képes minden szükséges információt megfelelő formában rögzíteni, 
ezenfelül a viszonyok objektum-orientált feldolgozása szükséges. Az SQLAlchemy automatikus 
sessionkezelést biztosít, ami jelentősen megkönnyíti a munkát, főleg a felhasználókezeléssel 
kapcsolatban. [8]

\SubSection{JavaScript és Vue}

A Brendon Eich által kifejlesztett JavaScript (JS) a webprogramozásban nagyon elterjedt
kliensoldali szkriptnyelv, nehéz megkerülni, mert a HTML csak egy jelölőnyelv. A JavaScript 
biztosítja a weboldalak interaktivitását is, mert alkalmas az eseményvezérelt programozásra. 
Multiparadigmás, just-in-time (JIT) compileres. Olyan kiegészítő technológiák, mint a JSON, 
a Jquery vagy az AJAX, mind elérhetőek beépítetten, továbbá olyan keretrendszereket is 
használhatunk, mint a Bootstrap vagy a Vue. Ez utóbbi nagyon hasznos volt a számomra, a 
felhasználói interfészek létrehozásával kapcsolatban, a Vue jobb altenatívát adott az 
űrlapokkal történő megoldás helyett. Az egyszerűség kedvéért nem is HTML, hanem Vue fájlokat
készítettem, bennük HMTL kódrésszel. Önmagában a JavaScript-et pedig a router 
(útvonalválasztó) fájl létrehozásához használtam. [8]

\Section{Felhasznált szoftverek}

A Python nyelven való munkához a JetBrains PyCharm fejlesztői alkalmazását használtam, mert kényelmetlen volt számomra a leginkább elterjedt Jupyter böngészőablakos használata. Az Oracle MySQL-jét használtam, mint adatbázis-kezelő szervert, mert az órarendgeneráló alkalmazás piacra kerülése esetén feltételezhetnénk, hogy az érdeklődés mértéke (felhasználók száma) szükségessé tenné szerver használatát. A webfejlesztéssel kapcsolatban egyrészt szintén a PyCharm-ra volt szükség, szerver létrehozásához és az adatbázis-kezeléshez (sessionkezeléshez), másrészt a vue fájlokhoz egy szövegszerkesztőre. Telepíteni kellett továbbá az npm-et, ami egy package manager. Feladata a függőségek (ilyen a webpack, az axios, sőt a vue is) és azok elfogadott verziószám tartományának számon tartása, melynek köszönhetően egyben tudunk telepíteni minden szükséges függőséget és később az upgradelés sem okoz problémát. Nem kell mindig egyenként vizsgálni, hogy milyen új verzió érhető el hozzájuk és egyenként upgradelni őket. Korábban a WebStorm-ot használtam, előfizetve rá egy hónap erejéig, de a konzulensem javasolt ingyenes szövegszerkesztőket, melyek megfelelőek lehetnek. Ezek közül a Notepad++ volt ismerős, logójában egy aranyos kaméleonnal, így ezt választottam. [8]

