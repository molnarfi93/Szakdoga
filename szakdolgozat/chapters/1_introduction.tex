\Chapter{Bevezetés}

Az iskolák és egyetemek a különböző osztályok, tanárok és tantermek órarendjének 
elkészítéséhez ma már számítógépes segítséget használnak, hagyományos algoritmusokat és a
mesterséges intelligencia egyik elterjedt eszközét, a genetikus algoritmust. Máskülönben 
nagyon időigényes, fáradtságos munka lenne összesen többszáz órarend ütközésmentes 
összehangolása, hiszen ugyanannak az osztálynak egy időben nem lehet több tanórája, egy tanár
sem tud két helyen egyszerre jelen lenni és egy tanteremben sem lehet két tanóra egy időben,
ott van továbbá a termek kapacitása, a tanárok heti óraszámai közti különbségek 
minimalizálása, az új osztályok képzésének szükségessége, pl. nyelvi vagy fakultációs órák 
miatt, és ez még nem minden. Munkám során olyan algoritmusokat írok, melyeknek köszönhetően 
adathalmaz (osztályok, tanárok, tantermek, tantárgyak) megadása után ütközésmentes és minden
feltételnek eleget tevő órarendeket tudunk generálni, majd grafikus felhasználói felületet 
hozok létre fölötte, mégpedig webeset.

