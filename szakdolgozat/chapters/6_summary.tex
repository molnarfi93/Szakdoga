\Chapter{Összefoglalás}

Az eltervezetteknek megfelelően implementáltam azokat az algoritmusokat, amelyek végül minden feltételt kielégítő órarendeket eredményeztek. Az ezek során szerzett tapasztalatokat ugyanúgy kifejtettem a szakdolgozatban, mint ahogyan elméleti elemzést is lefolytattam, ezáltal kézzelfogható eredményeket szolgáltatva a témakörben, melyek a későbbiekben nagyban felhasználhatóak lehetnek. A vizsgálatok egyik célpontja a genetikus algoritmus volt, ennek kapcsán kijelenthetem, hogy a genetikus algoritmust az időintervallum-hozzárendelések kapcsán érdemes használni, a többi részfeladat hagyományos alogritmusokkal való megoldását követően. A szakdolgozat elején számba vett, ismert órarendgeneráló alkalmazásokhoz képest az általam készített alkalmazás többet ad, a kora reggeli/késő délutáni órák számának minimalizálásával és a napi óraszámok arányos eloszlása/pénteki órák számának minimalizálásával, miközben egyben jó hatékonyságot érhetünk el a lyukas órák számának minimalizálása kapcsán is.

\newpage

\noindent {\huge{\textbf{Summary}}}

\bigskip

By right of the plan, I have implemented those algorithms, which resulted timetables finally, what satisfy every conditions. Experiences, what I gained during this, I have stated in the thesis, how I have conducted theoretical analysis too, so provided the profession with handable results, which can be usable in the future in large. Genetic algorithm was either target of investigations, about it I can assert, that usage of genetic algorithm is deserved at assignments of time intervals, after we solved the other subtasks by traditional algorithms. Compare to those famous timetable generator applications, which are presented in the beginning of thesis, the application is prepared by me add us more, with minimalization of number of early morning/late afternoon periods and with harmonic distribution of daily period numbers/minimalization of friday periods, while at the same time, we can reach good effectivity about minimalization of number of holes in periods, also.

%\huge{Irodalomjegyzék}
%
%\noindent [1]\quad $Pásztor E. - Obrony B.: Ökológia. Nemzeti Tankönyvkiadó Zrt., Budapest, 2007.$
%
%\noindent [2]\quad $Futó I.: Mesterséges intelligencia. Aula Kiadó, Budapest, 1999.$
%
%\noindent [3]\quad $Nagy T.: Operációkutatás. Miskolci Egyetemi Kiadó, Miskolc, 1998.$
%
%\textbf{Internetes források}
%
%\noindent [4]\quad $https://www.timetabler.com$
%
%\noindent [5]\quad $https://www.schedulebuilder.org$
%
%\noindent [6]\quad $https://www.primetimetable.com$
%
%\noindent [7]\quad $https://www.asctimetables.com$
%
%\noindent [8]\quad $https://www.github.com$
