\pagestyle{empty}

\noindent \textbf{\Large Adathordozó Használati útmutató}

\vskip 1cm

Az adathordozón megtalálható először is maga a szakdolgozat, pdf és tex fájlban is, utóbbi viszont önmagában nem életképes. Azok a kiszervezett fájlok, amelyek meghívásra kerülnek innen, a \textit{cover}, \textit{chapters}, \textit{styles} és \textit{images} mappákban találhatók. A \textit{application} mappa tartalmazza az elkészített órarendgeneráló alkalmazást és tartozékait. Két fő részből áll, mivel van a kliens oldal és a szerver oldal. Ezek különböző mappába kerültek, a harmadikban pedig az adatbázisba általam feltöltött adathalmaz van letárolva, hogy ne kelljen fáradnia a felhasználónak a webalkalmazáson keresztüli működés szemléletes kipróbálásához. Ezt az importálást MySQL Workbench grafikus felület esetén a \textit{Tools<Configuration<Restore connections} menüpontban lehet megtenni.

A kliensben a \textit{package.json} tartja számon a függőségeket, amiről a Felhasznált szoftverek alfejezetben beszéltem. Egyoldalas alkalmazásról beszélhetünk, melynek egységes header-jét az \textit{index.html} fájl tartalmazza, az üres body blokkba pedig az \textit{id} segítségével betöltődik az éppen aktuális vue fájl tartalma. A \textit{templates} mappában vannak azok a sablonok, amelyek a kigenerált órarendek adatainak html fájlba való beírásához szükségesek, illetve a regisztrációt követő vagy elfelejtett jelszó esetén kiküldött email-ekben szereplő adatok beírásához. Az \textit{src} mappában a következők találhatók:

\begin{itemize} 
	\item \textit{app.vue}: ez a fájl gondoskodik róla, hogy a különböző vue fájlok (komponensek) felcímkéződjenek a html fájlba való betöltéshez szükséges \textit{id}-val
	\item \textit{main.js}: ez a JavaScript fájl egy Vue objektumot példányosít, ezáltal fogjuk össze a vue komponenseinket
	\item \textit{router} mappa: ebben található az \textit{index.js} router (útvonalválasztó) fájl, ami tartalmazza a különböző vue komponensek beazonosításához, eléréséhez szükséges információkat
	\item \textit{components} mappa: a vue komponenseket tartalmazza, melyek funkcionálisan egy-egy html oldalnak feleltethetők meg és betöltésre várnak a webalkalmazás használata során
\end{itemize}

A szerver oldal Python fájlokat, valamint Python fájlokat tartalmazó mappákat tartalmaz. A \textit{timetable} mappában a \textit{timetable.py}, \textit{model.py} és \textit{database\_session.py} modulok vannak, az \textit{exercises} mappában pedig az órarendgeneráló algoritmusokat tartalmazó modulok, részfeladatonként egy, illetve külön mappában (\textit{class_definitions}) az osztálydefiníciók.

Mivel az alkalmazást kipróbálni szándékozó felhasználónak is a saját gépén szükséges szervert létrehoznia, telepíteni kellhet több dolgot:
\begin{itemize}
	\item \textit{npm}
	\item \textit{MySQL}
	\item \textit{PyCharm}
	\item \textit{Python kiegészítők:}
	\begin{itemize}
		\item \textit{Falcon}
		\item \textit{Waitress}
		\item \textit{JWT}
		\item \textit{Jinja}
		\item \textit{SQLAlchemy}
	\end{itemize}
\end{itemize}

Ha mindez megvan, regisztráció helyett javaslom a molnarfi93@gmail.com e-mail címmel és 666 jelszóval való belépést. Csak ebben az esetben férhetünk hozzá ugyanis az általam felvitt órarend-adathalmazhoz, a \textit{my\_timetables} komponensből.

